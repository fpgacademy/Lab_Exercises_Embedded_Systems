\documentclass[epsfig,10pt,fullpage]{article}

\newcommand{\LabNum}{1}
\newcommand{\CommonDocsPath}{../../common/docs}
\input{\CommonDocsPath/preamble.tex}

\begin{document}

\centerline{\huge Embedded Systems}
~\\
\centerline{\huge Laboratory Exercise \LabNum}
~\\
\centerline{\large Getting Started with Linux*}
~\\

\noindent
This is an introductory exercise in using a Linux* operating system on Intel's Cyclone\textsuperscript{\textregistered} V SoC 
devices. This discussion assumes that the reader is using the DE1-SoC development and 
education board, but the lab instructions apply equally well if other boards are being used,
such as the DE10-Standard or DE10-Nano. The DE-series boards are described on the 
{\it FPGAcademy.org} website.  Linux runs on the ARM* processor that is part 
of the Cyclone V SoC device. During the operating system boot process, Linux programs the 
Cyclone V FPGA with the \textit{DE1-SoC Computer} system (or similar system for other
boards). This system instantiates components inside the FPGA to make it easy to use the 
peripherals built into the DE1-SoC board, such as the LEDs, seven-segment displays, switches, 
and pushbuttons. For a detailed description of the computer system, please refer to 
a document such as \textit{DE1-SoC Computer System with ARM}, available as part of the
{\it Computer Organization} materials on {\it FPGAcademy.org}.

\section*{Part I}
\noindent
Read and complete Sections 1, 2, and 3 of the tutorial \textit{Using Linux on DE-series Boards}.
This tutorial is available as part of the {\it Embedded Systems} material on {\it FPGAcademy.org}.
These sections will guide you through setting up the Linux microSD card, running the 
Linux OS on the DE-series board, and communicating with the board from a host computer. 

\section*{Part II}
\noindent
Section 3.3 of the \textit{Using Linux on DE-series Boards} tutorial involves implementing a
program that increments the LEDR lights on the DE-series board. In this part you are to
write another program that controls the LED lights in a different manner, described below.

~\\
\noindent
Your program should turn on one LEDR light at a time. First, the rightmost light 
{\it LEDR}$_0$ should be on, then {\it LEDR}$_1$, then {\it LEDR}$_2$, and so on. When you
get to the leftmost light {\it LEDR}$_{9}$, the direction should be reversed. Only one 
LEDR light is ever on at one time. The effect should be a single light sweeping from 
right-to-left, then left-to-right, and so on.  
Use a delay so that the light moves at some reasonable speed. To implement the delay, you
can use a Linux system function, such as \texttt{nanosleep($\ldots$)}. Documentation for 
the \texttt{nanosleep} function can be found by typing \texttt{man nanosleep} in a 
Linux terminal, or by searching for it on the Internet.

\section*{Part III}
\noindent
Section 3.4 of the tutorial \textit{Using Linux on DE-series Boards} shows how to implement a 
device driver using interrupts for the KEY pushbuttons on the DE-series board. In this part you 
are to extend the functionality of that device driver.

~\\
\noindent
The existing KEY device driver initializes the LEDR port to $(\red{1}000000000)_2$, so that the 
leftmost light is on, and then increments this value whenever a KEY is pressed. You are to 
augment this code to display the count of pushbutton presses as a decimal digit, either on 
the seven-segment display {\it HEX0} or on the Linux Terminal window. You only need to 
display the decimal value corresponding to the four 
least-significant bits of the count, and wrap around to 0 when it exceeds 9. Always
leave the leftmost LEDR light set to 1 as a visual indicator that the device driver 
is present in the kernel. The count values displayed on the LEDR port should cycle through 
$\red{1}000000000, \red{1}00000000\red{1}, \red{1}00000000\red{1}0,
\red{1}00000000\red{11}, \ldots, \red{1}00000\red{1}00\red{1}, \red{1}000000000$, etc., and the 
corresponding digits displayed should be $\red{0}, \red{1}, \red{2}, \red{3}, \ldots, \red{9},
\red{0}$, 
respectively. 

~\\
\noindent
To display digits on the HEX0 seven-segment display, refer to the diagram given in 
Figure~\ref{fig:hex}. It shows the {\it Data} registers in the parallel ports for displays
HEX3$-$HEX0 and HEX5$-$HEX4. As shown in the figure, each segment in each display is
connected to a specific data-register bit. 

~\\
\noindent
If you are using the Linux Terminal window to display digits, then you may want to make 
use of some of the Terminal
window ASCII commands listed in Table 1. For example, the command \texttt{$\backslash$e[2J} clears 
the Terminal window, and the command \texttt{$\backslash$e[H} moves the Terminal {\it cursor} 
to the {\it home} position in the upper-left corner of the window.  In these 
commands \texttt{$\backslash$e} represents the \texttt{Escape} character. It can alternatively 
be specified by its ASCII code, for instance using the syntax \texttt{$\backslash$033}. You can send
such commands to the Terminal window by using the \texttt{printk} function. For example,
the Terminal window can be cleared by calling 

\begin{lstlisting}
printk (KERN_ERR "\e[2J");			// clear Terminal window
\end{lstlisting}

~\\
\noindent Documentation for the {\it printk} library function can be found by searching on
the Interet. Also, more Terminal window commands can be found by searching on the Internet for
\texttt{VT100 escape codes}.

~\\
\begin{figure}[H]
   \begin{center}
       \includegraphics{figures/fig_segment_port.pdf}
   \end{center}
   \caption{Diagram of the HEX display ports.}
	\label{fig:hex}
\end{figure}

\begin{table}[h]
\caption{Terminal window ASCII commands.}
~\\
\centering
\label{vt100}
\begin{tabular}{l|l}
		  {\bf Command} & {\bf Result} \\ \hline
		  \texttt{$\backslash$e[H} & move the cursor to the home position\\
		  \texttt{$\backslash$e[?25l} & hide the cursor \\
		  \texttt{$\backslash$e[?25h} & show the cursor \\
		  \texttt{$\backslash$e[2J} & clear window \\
		  \texttt{$\backslash$e[J} & clear window from cursor down \\ 
		  \texttt{$\backslash$e[1J} & clear window from cursor up \\
		  \texttt{$\backslash$e[ccm} & set foreground color to \texttt{cc}$^1$ \\
		  \texttt{$\backslash$e[yy;xxH} & set cursor location to row \texttt{yy}, column \texttt{xx}
		  
\end{tabular}
\end{table}

~\\
\noindent
Note 1: \texttt{cc} values are 31 (red), 32 (green), 33 (yellow), 34 (blue), 35 (magenta), 36
(cyan), and 37 (white)

~\\
\noindent

\noindent
\newpage
\input{\CommonDocsPath/copyright.tex}

\end{document}
