\documentclass[epsfig,10pt,fullpage]{article}

\newcommand{\LabNum}{7}
\newcommand{\CommonDocsPath}{../../common/docs}
\addtolength{\textwidth}{1.5in}
\addtolength{\oddsidemargin}{-0.75in}
\addtolength{\topmargin}{-0.75in}
\addtolength{\textheight}{1.5in}
\addtolength{\evensidemargin}{0.75in}
\setlength\parindent{0pt}
\raggedbottom

\usepackage{ae,aecompl}
\usepackage{epsfig,float,times}
\usepackage[hypcap]{caption}
\usepackage[pdftex, colorlinks]{hyperref}
\usepackage{graphicx}
\usepackage[usenames, dvipsnames]{color}
\usepackage{rotating}
\usepackage{tikz}
\usetikzlibrary{automata,positioning}
\usepackage{placeins}

\widowpenalty 10000
\clubpenalty 10000

\newcommand{\red}[1]{{\color{red}\sf{#1}}}
\newcommand{\green}[1]{{\color{green}\sf{#1}}}
\newcommand{\blue}[1]{{\color{blue}\sf{#1}}}
\definecolor{PineGreen}{rgb}{0.0, 0.47, 0.44}
\definecolor{ForestGreen}{rgb}{0.13, 0.55, 0.13}
\definecolor{Brown}{rgb}{0.59, 0.29, 0.0}

\newcommand{\UPDatePublished}{Oct 2021}
\newcommand{\versnum}{21.1} %version number quartus/AMP
\newcommand{\quartusname}{Quartus\textsuperscript{\textregistered} Prime}	
\newcommand{\UPTextBar}{For \quartusname{} \versnum{}}
\newcommand{\thisyear}{2021 } %for copyright
\newcommand{\company}{FPGAcademy.org}
\newcommand{\longteamname}{FPGAcademy.org}
\newcommand{\teamname}{FPGAcademy}
\newcommand{\website}{FPGAcademy.org}

\newcommand{\productAcronym}{AMP}
\newcommand{\productNameShort}{Monitor Program}

\newcommand{\productNameMedTM}{A Monitor Program}
\newcommand{\productNameMed}{A Monitor Program}

%\newcommand{\headerLogoFilePath}[1]{#1/FPGAcademy.png}

% listings is a package that supports encapsulating source code in LaTeX conveniently
\usepackage{listings}

\def\expandparam\lstinputlisting[#1]#2{\edef\tmp{\noexpand\lstinputlisting[#1]{#2}}\tmp}

%%%%%%%%%%%%%%%%%%%% Source Code Formatting %%%%%%%%%%%%%%%%%%%%
\definecolor{globalCommentColour}{rgb}{0.588,0.588,0.588}

%%%%%%%%%%%%%%%%%%%%%%%%%%%%%%%%%%%%%%%%%%%%%%%%%%%%
% Defining language style
% NiosII ASM
\lstdefinelanguage[NiosII]{Assembler} {
  morekeywords={add, addi, and, andhi, andi, beq, bge, bgeu, bgt, bgtu, ble,  bleu, blt, bltu, bne, br, break,
  bret, call, callr, cmpeq, cmpeqi, cmpge, cmpgei, cmpgeu, cmpgeui, cmpgt, cmpgti, cmpgtu, cmpgtui, cmple,
  cmplei, cmpleu, cmpleui, cmplt, cmplti, cmpltu, cmpltui, cmpne, cmpnei, custom, div, divu, eret, flushd,
  flushda, flushi, flushp, initd, initda, initi, jmp, jmpi, ldb, ldbio, ldbu, ldbuio, ldh, ldhio, ldhu, ldhuio,
  ldw, ldwio, mov, movhi, movi, movia, movui, mul, muli, mulxss, mulxsu, mulxuu, nextpc, nop, nor, or, orhi, ori,
  rdctl, rdprs, ret, rol, roli, ror, sll, slli, sra, srai, srl, srli, stb, stbio, sth, sthio, stw, stwio,
  sub, subi, sync, trap, wrctl, wrtcl, wrprs, xor, xori, xorhi, xori},
  morekeywords=[2]{.abort, .ABORT, .align, .app-file, .ascii, .asciz, .balign, .byte, .comm, .data, .def,
  .desc, .dim, .double, .eject, .else, .end, .endef, .endif, .equ, .equiv, .err, .extern, .file, .fill, .float,
  .global, .globl, .hword, .ident, .if, .include, .int, .irp, .irpc, .lcomm, .lflags, .line, .linkonce, .ln,
  .list, .long, .macro, .mri, .nolist, .octa, .org, .p2align, .psize, .quad, .rept, .sbttl, .scl, .section,
  .set, .short, .single, .size, .sleb128, .skip, .space, .stadb, .stabn, .stabs, .string, .symver, .tag,
  .text, .title, .type, .val, .uleb128, .word},
  morekeywords=[3]{et, bt, gp, sp, fp, ea, sstatus, ra, pc, status, estatus, bstatus, ienable, ipending, cpuid,
  exception, pteaddr, tlbacc, tlbmisc, eccinj, badaddr, config, mpubase, mpuacc},
  sensitive=t,
  alsoletter=.,
  morestring=[b]",
  morecomment=[s]{/*}{*/},
  morecomment=[l]\#,
}[keywords,comments,strings]
   
%% NOTE: morekeywords=[2] are GNU directives.
   
\definecolor{niosInstructionColour}{rgb}{0.000,0.608,0.000}
\definecolor{niosDirectiveColour}{rgb}{0.000,0.000,0.902}
\definecolor{niosSpecialRegColour}{rgb}{0.000,0.000,0.000}
\definecolor{niosStringColour}{rgb}{0.808,0.482,0.000}
   
%% NOTE: To make bold use: =\bfseries\color{<colour>}
\lstdefinestyle{defaultNiosStyle} {
  language=[NiosII]{Assembler},
  stringstyle=\color{niosStringColour},
  keywordstyle=\color{niosInstructionColour},
  keywordstyle=[2]\color{niosDirectiveColour},
  keywordstyle=[3]\itshape\color{niosSpecialRegColour}
}
%%%%%%%%%%%%%%%%%%%%%%%%%%%%%%%%%%%%%%%%%%%%%%%%%%%%

%%%%%%%%%%%%%%%%%%%%%%%%%%%%%%%%%%%%%%%%%%%%%%%%%%%%
% Defining language style
% ArmA9 ASM
\lstdefinelanguage[ArmA9]{Assembler} {
  morekeywords={ADC, ADD, ADDS, AND, ANDS, B, BAL, BEQ, BGE, BGT, BL, BLT, BIC, BKPT, BLX, BNE, BX, CDP, CLZ, CMN, CMP, EOR,
  EORS, LDC, LDM, LDR, LDRB, LDRBT, LDRH, LDRSB, LDRSH, LDRT, LSL, MCR, MLA, MOV, MOVW, MOVT, MRC, MRS, MSR, MUL, MVN, ORR, PLD,
  ROR, RSB, RSC, SBC, SMLAL, SMULL, STC, STM, STR, STRB, STRBT, STRH, STRT, SUB, SUBS, SWI, SWP, SWPB, TEQ, UMLAL,
  PUSH, POP, MOVS, RORS, LSR},
  morekeywords=[2]{.abort, .ABORT, .align, .app-file, .ascii, .asciz, .balign, .byte, .comm, .data, .def,
  .desc, .dim, .double, .eject, .else, .end, .endef, .endif, .equ, .equiv, .err, .extern, .file, .fill, .float,
  .global, .globl, .hword, .ident, .if, .include, .int, .irp, .irpc, .lcomm, .lflags, .line, .linkonce, .ln,
  .list, .long, .macro, .mri, .nolist, .octa, .org, .p2align, .psize, .quad, .rept, .sbttl, .scl, .section,
  .set, .short, .single, .size, .sleb128, .skip, .space, .stadb, .stabn, .stabs, .string, .symver, .tag,
  .text, .title, .type, .val, .vectors, .uleb128, .word},
  morekeywords=[3]{SP, PC, MIDR, CTR, TCMTR, TLBTR, MPIDR, ID_PFR0, ID_PFR1, ID_DFR0, ID_MMFR0, ID_MMFR1, ID_MMFR2,
  ID_MMFR3, ID_ISAR0, ID_ISAR1, ID_ISAR2, ID_ISAR3, ID_ISAR4, CCSIDR, CLIDR, AIDR, CSSELR, TTBR0, TTRB1, TTBR2, DACR,
  DFSR, IFSR, ADFSR, AIFSR, DFAAR, IFAR, ICIALLUIS, BPIALLIS, PAR, ICIALLU, ICIMVAU, BPIALL, DCIMVAC, DCISW, V2PCWPR,
  DCCVAC, DCCSW, DDIMVAC, DCISW, TLBALLIS, TLBIMVAIS, TLBIASIDIS, TLBIMVAAIS, TLBIALL, TLBIMVA, TLBIASID, TLBIMVAA,
  PMCR, PMCNTENSET, PMCNTENCLR, PMOVSR, PMSWINC, PMSELR, PMXEVTYPER, PMXEVCNTR, PMUSERENR, PMINTENSET, PMINTENCLR,
  PRRR, NRRR, PLEIDR, PLEASR, PLEFSR, PLEUAR, PLEPCR, VBAR, MVBAR, ISR, FCSEIDR, CONTEXTIDR, TPIDRURW, TPIDRURO, TPIDRPRW},
  sensitive=f,
  alsoletter=.,
  morestring=[b]",
  morecomment=[s]{/*}{*/},
  morecomment=[l]{//},
}[keywords,comments,strings]
   
%% NOTE: morekeywords=[2] are GNU directives.
   
\definecolor{armInstructionColour}{rgb}{0.000,0.608,0.000}
\definecolor{armDirectiveColour}{rgb}{0.000,0.000,0.902}
\definecolor{armSpecialRegColour}{rgb}{0.000,0.000,0.000}
\definecolor{armStringColour}{rgb}{0.808,0.482,0.000}
   
\lstdefinestyle{defaultArmStyle} {
  language=[ArmA9]{Assembler},
  stringstyle=\color{armStringColour},
  keywordstyle=\color{armInstructionColour},
  keywordstyle=[2]\color{armDirectiveColour},
  keywordstyle=[3]\itshape\color{armSpecialRegColour}
}
%%%%%%%%%%%%%%%%%%%%%%%%%%%%%%%%%%%%%%%%%%%%%%%%%%%%

%%%%%%%%%%%%%%%%%%%%%%%%%%%%%%%%%%%%%%%%%%%%%%%%%%%%
% Defining language style
% FPGAcademy ASM
\lstdefinelanguage{ASM}{
  morekeywords = [1]{mv, mvt, mvne, mvcc, add, sub, st, ld, and, b, bne, beq, bcc, bcs},
  morekeywords = [2]{word, define},
  keywordstyle = [1]\color{ForestGreen},
  keywordstyle = [2]\color{blue},
  sensitive = true,
  morecomment = [l]{//},
}

\lstset{
  language = ASM,
  basicstyle=\small\color{black}\ttfamily,
  commentstyle=\small\color{Brown}\itshape\ttfamily,
  showstringspaces=false,
  frame=none, %lines % boxed listings
  breaklines=true,
  breakatwhitespace=true,
  tabsize=3
}
%%%%%%%%%%%%%%%%%%%%%%%%%%%%%%%%%%%%%%%%%%%%%%%%%%%%

%%%%%%%%%%%%%%%%%%%%%%%%%%%%%%%%%%%%%%%%%%%%%%%%%%%%
% Defining language style
% Java
\definecolor{javaStringColour}{rgb}{0.808,0.482,0}
%%%%%%%%%%%%%%%%%%%%%%%%%%%%%%%%%%%%%%%%%%%%%%%%%%%%

%%%%%%%%%%%%%%%%%%%%%%%%%%%%%%%%%%%%%%%%%%%%%%%%%%%%
% Defining language style
% C
\definecolor{CStringColour}{rgb}{0.808,0.482,0}

\lstset{
  language = C,
  basicstyle=\small\color{black}\ttfamily, 
  commentstyle=\small\color{PineGreen}\itshape\ttfamily,
  keywordstyle=\small\color{blue}\bfseries\ttfamily,
  showstringspaces=false,
  frame=none, %lines % boxed listings
  breaklines=true,
  breakatwhitespace=true,
  tabsize=3
}
%%%%%%%%%%%%%%%%%%%%%%%%%%%%%%%%%%%%%%%%%%%%%%%%%%%%

%%%%%%%%%%%%%%%%%%%%%%%%%%%%%%%%%%%%%%%%%%%%%%%%%%%%
% Defining language style
% Verilog
\definecolor{verilogCommentColour}{rgb}{0.000,0.502,0.000}

\lstdefinestyle{defaultVerilogStyle} {
  language={Verilog},
  keywordstyle=\color{blue},
  commentstyle=\color{verilogCommentColour}
}
%%%%%%%%%%%%%%%%%%%%%%%%%%%%%%%%%%%%%%%%%%%%%%%%%%%%

%%%%%%%%%%%%%%%%%%%%%%%%%%%%%%%%%%%%%%%%%%%%%%%%%%%%
% Defining language style
% VHDL
\lstdefinestyle{defaultVHDLStyle} {
  language={VHDL},
  keywordstyle=\color{blue},
  commentstyle=\color{verilogCommentColour}
}
%%%%%%%%%%%%%%%%%%%%%%%%%%%%%%%%%%%%%%%%%%%%%%%%%%%%

%%%%%%%%%%%%%%%%%%%%%%%%%%%%%%%%%%%%%%%%%%%%%%%%%%%%
% Defining language style
% LaTeX
\lstdefinelanguage[LocalLaTeX]{TeX}[LaTeX]{TeX}{moretexcs={bf, it, sf, lstset},}

\lstdefinestyle{defaultLocalLatexStyle} {
  language=[LocalLatex]{TeX},
  keywordstyle=\color{blue}\bfseries,
  keywordstyle=[2]\color{blue},
  keywordstyle=[3]\color{blue}\bfseries
}
%%%%%%%%%%%%%%%%%%%%%%%%%%%%%%%%%%%%%%%%%%%%%%%%%%%%

%%%%%%%%%%%%%%%%%%%%%%%%%%%%%%%%%%%%%%%%%%%%%%%%%%%%
% Defining language style
% Default
\lstset{
  basicstyle=\small\color{black}\ttfamily,
  commentstyle=\small\color{globalCommentColour}\itshape\ttfamily,
  keywordstyle=\small\color{blue}\bfseries\ttfamily,
  showstringspaces=false,
  frame=none, %lines % boxed listings
  breaklines=true,
  breakatwhitespace=true,
  tabsize=3
}
%%%%%%%%%%%%%%%%%%%%%%%%%%%%%%%%%%%%%%%%%%%%%%%%%%%%


\hypersetup{
  pdftitle={Embedded Systems Lab Exercise \LabNum},
  linkcolor=blue,
  hyperindex=true,
  pdfauthor={FPGAcademy.org},
  pdfkeywords={FPGAcademy.org, FPGAcademy, Lab, Exercise, Embedded Systems},
  bookmarks,
  bookmarksopen=false,
  filecolor=blue,
  pdfstartview={FitH},
  urlcolor=blue,
  plainpages=false,
  pdfpagelabels=true,
  linkbordercolor={1 1 1} %no color for link border
}



\begin{document}

\centerline{\huge Embedded Systems}
~\\
\centerline{\huge Laboratory Exercise \LabNum}
~\\
\centerline{\large Using the ADXL345 Accelerometer}
~\\

\noindent
The purpose of this exercise is to experiment with the ADXL345 accelerometer chip that is
included on a number of DE-series board. In this writeup we will assume that you are using 
the DE1-SoC board, but the same instructions apply to other DE-series board, such as the DE10-Nano
and DE10-Standard. You will write programs that interact with the ADXL345 device under 
the Linux* operating system, running on the ARM* processor in the DE1-SoC Computer. Also, 
you will add support in the Linux kernel for the ADXL345 chip by creating a character device
driver.

\noindent
\section*{Part I}

\noindent
To learn about the operation of the ADXL345 accelerometer chip, read the tutorial called
{\it Using the Accelerometer in DE-SoC Boards}. This tutorial is available on the 
FPGAcademy.org website. An example program that demonstrates use of the accelerometer
is included with the tutorial.  This program assumes that no operating system is
being used (i.e., a ``bare metal'' environment), and therefore uses simple dereferencing of 
pointers to communicate with the device. But, since your code needs to run under Linux on the
DE1-SoC Computer, you will have to map the required physical addresses into virtual addresses
by using Linux kernel functions like {\it mmap} and the {\it /dev/mem} interface.
These concepts are discussed in detail in the tutorial {\it Using Linux on the DE1-SoC}.

~\\
\noindent
Perform the following:

\begin{enumerate}
\item Read through the tutorial {\it Using the Accelerometer in DE-SoC Boards}. Create a new
version of the C program that accompanies the tutorial, which uses virtual addresses and
can run under Linux.
\item In your program, configure the ADXL345 device to provide an output data rate of 12.5
Hz, using a fixed 10-bit resolution and +/- 16 g range. These settings provide an appropriate
level of sensitivity from the ADXL345 device, of about 31 mg per least-significant bit.
\item Use a Linux Terminal to compile and run your program. Tilt the DE1-SoC board in various
directions and observe the acceleration data that is printed by your program. Since the z axis
of the accelerometer chip on the DE1-SoC board is aligned with earth's gravity, your program
should read acceleration of approximately 1,000 mg in the z direction when the DE1-SoC board
is stationary. You should end up with a program that prints out acceleration data in the format 
\texttt{<X-axis acceleration in mg>, <Y-axis acceleration in mg>, <Z-axis acceleration in mg>}.
\end{enumerate}

\noindent
\section*{Part II}

\noindent
In Part I you wrote a user-level program to directly read/write registers in the ADXL345
device by using memory-mapped I/O. In this part, you will use a different approach, by
adding support to the Linux kernel for accessing the device.
You are to create a Linux character device driver, in the form of a kernel
module, that provides a file-based I/O interface to accelerometer data. The driver must create
the file {\it /dev/accel} in the Linux filesystem. A read of this file should return accelerometer
data in the format \texttt{R XXXX YYYY ZZZZ SS}, where {\it R} is 1 if new accelerometer data
is being provided, {\it XXXX}, {\it YYYY}, and {\it ZZZZ} are acceleration data in the 
x, y, and z axes, and {\it SS} is the scale factor in mg/LSB for the acceleration data. As
an example, if the ADXL345 has new data to report, then a read of the file might return:
"1 0 -1 32 31", which would represent 0 mg acceleration in the x axis, -31 mg acceleration in
the y axis, and 992 mg acceleration in the z axis. If you were to perform another read from {\it
/dev/accel} immediately, then the device might not be ready to provide new data; it would
then respond with "0 0 -1 32 31", indicating old data.

~\\
\noindent
An outline of the required code for the accelerometer device driver is given in 
Figure~\ref{fig:accel}. Lines \ref{line:includes} to~\ref{line:includes2} include header files
that are needed for the driver. Global variables that are used to access the accelerometer,
which will be described later, are declared in lines~\ref{line:dec1} and~\ref{line:dec2}.
Line~\ref{line:vars} is a placeholder for the declarations of function prototypes and variables
that are needed for the character device driver. Prototypes have to be declared for the functions
that are executed when opening, reading, and closing the driver. Variables of type 
\texttt{dev\_t}, \texttt{cdev}, and \texttt{class} have to be declared and initialized in
the function {\it start\_accel}, shown in Lines~\ref{line:start1} to~\ref{line:start2},
which is executed when the accelerometer driver is inserted into the Linux kernel. Refer to
Exercise 3 for a more detailed discussion about the functions and variables that are needed for
character device drivers.

~\\
\noindent
To provide the kernel module with access to the {\it I2C controller}, which is used to
communicate with the ADXL345 device, line~\ref{line:c1} calls the \texttt{ioremap\_nocache}
function. Information about this function can be found in the tutorial {\it Using Linux on
the DE1-SoC}. Line~\ref{line:c2} computes a virtual address for the {\it system manager}, 
which has to be configured for use with the ADXL345 device. The next three lines call the
\texttt{Pinmux\_Config}, \texttt{I2C0\_Init}, and \texttt{ADXL345\_Init} functions. You
should be able to reuse the code that you wrote for these functions in Part I of this exercise.
You should initialize the ADXL345 with the same settings as for Part I.

~\\
Line~\ref{line:ADXL345} is a placeholder for the various functions that are needed to read
and write to registers in the ADXL345 device. You should be able to reuse the code from 
Part I for these functions. The remainder of the code in Figure~\ref{fig:accel}
refers to the functions that are needed to open, close, and read from the character device driver.
A detailed discussion of these functions can be found in Laboratory Exercise 3.

~\\
\noindent
Perform the following:

\begin{enumerate}
\item Create a file named {\it accel.c} and fill in the missing code from Figure~\ref{fig:accel}.
Using a Makefile, compile your code to create the 
kernel module {\it accel.ko} and insert this module into the Linux kernel. 
\item Test your device driver by executing a command such as \texttt{cat /dev/accel} using a
Linux Terminal window. Try tilting the board in various directions and confirm that the
driver provides appropriate results each time that you read from {\it /dev/accel}.
\item
Once you have confirmed correct operation of your character device driver, write a user-level
program called {\it part2.c} that uses the driver.
A skeleton of an example program is given in Figure~\ref{fig:part2}. Fill in the rest 
of the code so that it performs some operations using the device driver. For example, you could 
write a loop that continuously reads from the device and prints the results whenever new
data is available. Compile your program using a command such as \texttt{gcc -Wall -o part2
part2.c}, and test the program.
\end{enumerate}

\noindent
\section*{Part III}

In Part II your character device driver supports only read operations. For this part, you
are to add the following commands that can be written to your device driver:

\begin{itemize}
		  \item \texttt{device}: prints on the Terminal (using {\it printk}) the ADXL345 device ID.
		  \item \texttt{init:} re-initializes the ADXL345
		  \item \texttt{calibrate:} executes a calibration routine
		  \item \texttt{format F G:} sets the data format to fixed 10-bit resolution $(F = 0)$, or
					 full resolution $(F = 1)$, with range {\it G} = +/- $2, 4, 8,$ or 16 g
		  \item \texttt{rate R:} sets the output data rate to $R$ Hz. Your code should support
					 a few examples of data rates, such as 25 Hz, 12.5 Hz, and so on.
\end{itemize}

\noindent
Perform the following:

\begin{enumerate}
\item Augment your device-driver code from Part II to support {\it write} operations, 
and implement the commands listed above. 
\newpage
\lstset{language=C,numbers=left,escapechar=|}
\begin{figure}[h]
\begin{center}
\begin{minipage}[t]{15 cm}
\begin{lstlisting}[name=dots]
|\label{line:includes}|#include <linux/module.h>
#include <linux/kernel.h>
#include <linux/fs.h>
#include <linux/cdev.h>
#include <linux/device.h>
#include <asm/io.h>
|\label{line:includes2}|#include <asm/uaccess.h>

// Declare global variables needed to use the accelerometer
|\label{line:dec1}|volatile int * I2C0_ptr;	// virtual address for I2C communication
|\label{line:dec2}|volatile int * SYSMGR_ptr;	// virtual address for System Manager communication

// Declare other variables and prototypes needed for a character device driver
|\label{line:vars}$\ldots$| code not shown

/* Code to initialize the accelerometer driver */
|\label{line:start1}|static int _|$\,$|_init start_accel(void)
{
	// initialize the dev_t, cdev, and class data structures
	|$\ldots$| code not shown

	// generate virtual addresses
	|\label{line:c1}|I2C0_ptr = ioremap_nocache (0xFFC04000, 0x00000100);
	|\label{line:c2}|SYSMGR_ptr = ioremap_nocache (0xFFD08000, 0x00000800);
	if ((I2C0_ptr == 0) |$\mid\,\mid$| (SYSMGR_ptr == NULL))
		printk (KERN_ERR "Error: ioremap_nocache returned NULL\n");

	Pinmux_Config ();
	I2C0_Init ();
	ADXL345_Init ();
	return 0;
|\label{line:start2}|}

// Functions needed to read/write registers in the ADXL345 device
|\label{line:ADXL345}||$\ldots$| code not shown

static void _|$\,$|_exit stop_accel(void)
{
	/* unmap the physical-to-virtual mappings */
	iounmap (I2C0_ptr);
	iounmap (SYSMGR_ptr);

	/* Remove the device from the kernel */
	|$\ldots$| code not shown
}

// Code for device_open and device_release
|$\ldots$| code not shown

static ssize_t device_read(struct file *filp, char *buffer, size_t length, loff_t *offset)
{
	|$\ldots$| code not shown
}
\end{lstlisting}
\end{minipage}
\caption{An outline of the ADXL345 device driver code.}
\label{fig:accel}
\end{center}
\end{figure}
\clearpage
\item Write a user-level program, called {\it part3.c}, that uses your character device driver.
Implement a simple graphical demo that draws a ``bubble'' on the screen. When the DE1-SoC
board is stationary the bubble should appear in the middle of the screen. Tilting the
board right, left, up, or down should cause the bubble to ``move'' accordingly.
Also, print in the top-left of the screen the values of x, y, and z acceleration each time
new data is available from your device driver.
\item To implement graphics, you can use either ASCII commands in a Terminal window, or you can
connect a VGA monitor to the DE1-SoC board and use VGA graphics.  ASCII graphics were
described in Lab Exercise 5, and VGA graphics were discussed in Lab Exercise 6.
\item Experiment by writing different commands to {\it /dev/accel} and examining the effect
on your graphics display. Depending on the data format that you set in the accelerometer,
the graphics displayed by your program may appear ``jittery''. To ameliorate this effect,
you can compute a running average of the acceleration data. For example, to compute an
average acceleration for the x-axis, an appropriate calculation could be
\lstset{language=C,numbers=none}
\begin{lstlisting}
av_x = av_x * |$\alpha$| + accel_x * |$(1 - \alpha)$|
\end{lstlisting}
Here, {\it av\_x} is the average acceleration, and {\it accel\_x} is the most recent
x-axis data provided by the ADXL345. Smaller/larger values of $\alpha$ can be used to
decrease/increase the effect of new data on the average.
\end{enumerate}

~\\
\lstset{language=C,numbers=none,escapechar=|}
\begin{figure}[h]
\begin{center}
\begin{minipage}[t]{13.5 cm}
\begin{lstlisting}[name=part1]
#include <stdio.h>
#include <string.h>
#include <errno.h>
#include <fcntl.h>
#include <unistd.h>

// number of characters to read from /dev/accel
#define accel_BYTES 20

int main(int argc, char *argv[])
{
	int accel_FD;						// file descriptor
	char accel_buffer[accel_BYTES];		// buffer for accel char data
	// Declare other variables
	|$\ldots$| code not shown 
    
	// Open the character device driver
  	if ((accel_FD = open("/dev/accel", O_RDWR)) == -1)
	{
		printf("Error opening /dev/accel: %s\n", strerror(errno));
		return -1;
	}
	
	// Perform some operations using the ADXL345 device
	|$\ldots$| code not shown

	close (accel_FD);
	return 0;
}
\end{lstlisting}
\end{minipage}
\caption{A program that communicates with /{\it dev}/{\it accel}.}
\label{fig:part2}
\end{center}
\end{figure}
\newpage
\noindent
\section*{Part IV}

\noindent
In addition to providing acceleration data, the ADXL345 device also supports ``tap'' and
``double-tap'' detection. Perform the following:

\begin{enumerate}
\item Add tap and double-tap detection to your device-driver code from Part III. To learn
how to configure the accelerometer for tap detection, consult the product {\it Data Sheet},
which is available on Analog Device's website ({\it www.analog.com}). Some recommended 
tap-detection settings are

\begin{tabular}{ l l }
	\\Tap threshold:& ~~~~~~~~3 g \\
	Maximum tap duration:& ~~~~~~~~0.02 seconds\\
	Tap latency:& ~~~~~~~~0.02 seconds \\
	Double-tap window:& ~~~~~~~~0.3 seconds\\
\end{tabular}

\item Modify the read function for your character device driver so that it returns data in
the format \texttt{HH XXXX YYYY ZZZZ SS}. Here, {\it HH} is a two-digit hexadecimal value
corresponding to the contents of the ADXL345 {\it Sources of Interrupt} register. In addition to
reporting when new acceleration data is available this register also indicates when tap
or double-tap actions have been detected.
\item Write a user-level program, called {\it part4.c}, that demonstrates the detection of 
tap and double-tap actions. One possibility is to augment the user-level program from Part III
such that the animation displayed changes based on whether a tap or double-tap has occurred.
\end{enumerate}

\vskip 0.8in
\noindent
\newpage
%%%%%%%%%%%%%%%%%%%%%%%%%%%%%%%%%%%%%%%%
%%% FPGAcademy Copyright Information %%%
%%%%%%%%%%%%%%%%%%%%%%%%%%%%%%%%%%%%%%%%

%Always put the copyright on a new page (clear page), with some vertical space from top
\clearpage
\vspace{1in}

\noindent

Copyright {\copyright} FPGAcademy.org. All rights reserved. FPGAcademy and the 
FPGAcademy logo are trademarks of FPGAcademy.org.  This document is provided 
"as is", without warranty of any kind, express or implied, including but not 
limited to the warranties of merchantability, fitness for a particular purpose 
and noninfringement. In no event shall the authors or copyright holders be 
liable for any claim, damages or other liability, whether in an action of 
contract, tort or otherwise, arising from, out of or in connection with the 
document or the use or other dealings in the document.
~\\
~\\
**Other names and brands may be claimed as the property of others.



\end{document}
